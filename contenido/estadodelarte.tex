\section{Estado del arte}\label{sec:soa}

En los últimos años ha habido un crecimiento en el interés de los sistemas de recomendación \cite{Abdomavicius}, desde la aparición de los primeros artículos al respecto a mediados de los años 90 \cite{resnick}. Por ejemplo, en \cite{nageswara} puede encontrarse un inventario de sistemas de recomendación existentes que han sido desarrollados tanto en un ámbito académico como en la industria.\\

En la creación de sistemas de recomendación hay dos grandes corrientes, que son las explicadas en el \autoref{chap:recom}. Existen motores de recomendación que se basan en atributos del contenido y sistemas de recomendación que se basan en opiniones e interacciones entre usuarios (filtrado colaborativo). Los últimos, normalmente requieren de una gran cantidad de datos, por lo que es habitual que grandes compañías como Amazon \cite{Amazon} o Google \cite{Google} las utilicen habitualmente. Además de requerir grandes cantidades de datos, este tipo de sistemas tienen el problema de que los elementos nuevos son difícilmente recomendables al carecer de opiniones. Los sistemas basados en contenido no tienen este problema, ya que la recomendación se efectúa, principalmente, de acuerdo con los metadatos de cada elemento. De esta forma, un elemento, por nuevo que sea, será recomendado siempre y cuando se hayan introducido los metadatos. Como nota adicional, hay que tener en cuenta que normalmente no se explica el funcionamiento de los mejores sistemas de recomendación, ya que hay compañías para las que una gran parte del negocio depende de estos sistemas.\\

En el filtrado colaborativo, el sistema recoge como entrada un conjunto de puntuaciones de los usuarios sobre una serie de items. Los usuarios pueden compararse comparando en puntuaciones compartidas o similares. Los items pueden compararse de similar manera con usuarios con votaciones similares. La puntuación hipotética de un usuario sobre un item puede predecirse usando información de usuarios de su vecindad y películas de la vecindad de la película a predecir. Dentro de los sistemas que usan filtrado colaborativo, pueden distinguirse: \textbf{filtros basados en usuarios} \cite{resnick}, donde la predicción se consigue mediante las puntuaciones de ese item dadas por usuarios similares. En este caso, se hace necesario definir una medida de similaridad entre usuarios. \textbf{Filtros basados en items}, en los que la predicción se usa buscando items similares \cite{Amazon}, \cite{sarwar}, \cite{karypis}. \textbf{Filtros basados en modelos}, los modelos anteriormente mencionados tienen una complejidad computacional cuadrática \cite{Candiller}, lo cual los hace inviables para conjuntos de datos muy grandes y con necesidades rápidas de respuesta. Estos problemas son lo que tratan de solucionar los filtros de este último tipo \cite{breese}. Estos modelos utilizan conceptos como el clustering para reducir la complejidad del problema, agregando los usuarios en clusters independientes que permitan reducir la complejidad \cite{oconnor}.\\

En los sistemas de recomendación basados en contenido, se realiza una recomendación al usuario usando como base una descripción del item y un perfil del usuario. Estos sistemas requieren de una forma de describir los items que pueden ser recomendados, una forma de modelar al usuario y una forma de comparar items. El perfil de usuario puede ser construido de forma implícita o explícita, a base de cuestioanrios. En nuestro caso, como se verá más adelante, se construirá de forma implícita a través de atributos del item dado por el usuario. En este tipo de filtros, como hemos dicho, es necesario poder comparar elementos textuales entre sí. Existen muchas formas de hacerlo, como por ejemplo mediante la distancia google \cite{cilibrasi} que infiere similaridades mediante la coocurrencia de estos elementos en páginas web. Sin embargo, esta medida tiene el problema de requerir de un servicio externo como es la API de GOogle para poder calcularla. En este proyecto, utilizaremos una distancia definida por la presencia de keywords compartidas (pasando antes por un preprocesado de las mismas), director y actores principales. Un problema que tienen en ocasiones los sistemas basados en contenido, es la sobreespecializacion \cite{zhang}. Es decir, que recomiendan elementos demasiado similares a los dados, lo que puede dar lugar a una falta de originalidad. Sin embargo, consideramos que no hay películas tan similares como para dar lugar a este problema.\\

Los sistemas de recomendación basados en filtrado colaborativo tienen un problema bastante importante, que se tratará en detalle más adelante en la memoria. Este problema es el llamado \textit{sparsity} en inglés. Esto es, que la mayoría de usuarios únicamente han votado una parte muy pequeña de las películas, por lo que la matriz que contiene las películas y las puntuaciones de cada usuario tiene una gran cantidad de ceros. En el Premio Netflix \cite{netflix} la solución que se propuso fue realizar una factorización de matrices, de forma que se condensa esta información e incluso se pueden obtener representaciones de las películas en un sistema de coordenadas equiparable a diferentes géneros. En \cite{ilhami2014film} se realiza una implementación de un sistema de recomendación basado en filtrado colaborativo, aunque como puede verse en el artículo, cuentan con una cantidad de datos mucho mayor. En la \autoref{tab:compareSystems} se realiza una comparación entre el sistema propuesto en este artículo y el propuesto en este proyecto.

\begin{table}[H]
\centering
%\resizebox{0.8\textwidth}{!}{%
\begin{tabular}{p{0.3\linewidth}p{0.3\linewidth}p{0.3\linewidth}}
\hilne \textbf{Aspecto} & \textbf{Sistema de este TFE} & \textbf{Sistema propuesto en \cite{ilhami2014film}} \\ \hline 
Recomendaciones dependientes del usuario & El sistema recomienda películas independientemente del usuario que las solicita.                                                                                & Cada usuario está modelado de forma diferente, por lo que las recomendaciones sí que son únicas                                                                                                    \\
Nuevos usuarios y elementos              & Al ser independiente del usuario, la aparicion de nuevos usuarios no es un problema. Los nuevos contenidos serán recomendados una vez se rellenen sus metadatos & Puede haber problemas al tener usuarios que no hayan valorado ninguna película o películas que no hayan sido recomendadas. Es el problema denominado \textit{cold start}                           \\
Posibilidad de indexación                & Podemos tener precalculadas las recomendaciones, lo cual supone un gran ahorro de tiempo a la hora de servir las recomendaciones a los usuarios                 & Tener las recomendaciones precalculadas es muy costoso, al depender del producto entre usuarios y películas.                                                                                       \\
Agnosticismo                             & El sistema recomienda las películas independientemente de la plataforma en la que se encuentran, al contrario que sistemas de Netflix o Amazon                  & El conjunto de películas utilizado no proviene de una plataforma en cocnreto. Sin embargo, al usar votos de usuarios, es complicado introducir una película que no esté en la plataforma de votos. \\
Implementación Real                      & El proyecto de este trabajo se ha desarrollado de una forma \textit{end to end}, de forma que se ha generado un sistema de recomendación usable                 & El sistema de recomendacion creado es de laboratorio, no hay posibilidad de usarlo                                                                                                                
\end{tabular}%
%}
\caption{Comparación entre el sistema de recomendación propuesto en \cite{ilhami2014film} y en esta memoria. El del artículo es un sistema basado en filtrado colaborativo y factorización de matrices, mientras que el de este trabajo es un filtro basado en contenido, con una heurística posterior de puntuación de las películas. Como puede verse, hay grandes diferencias entre ambos sistemas, destacando el de este proyecto por no necesitar tantos datos para funcionar, poder incluir nuevos elementos fácilmente y además estar implementado en un entorno real como el chatbot de Telegram.}
\label{tab:compareSystems}
\end{table}

La evaluación de los sistemas de recomendación es también un punto fundamental en este área. En la mayoría de los casos, la forma de realizar esta evaluación es comparando puntuaciones predecidas con las que se tienen originalmente, teniendo un conjunto de usuarios de entrenamiento y otro de test. Aunque esta información es relevante e interesante, en los sistemas de recomendación es fundamental evaluar la relevancia de las recomendaciones realizadas \cite{herlocker}, ya que aunque la predicción puede ser buena, puede estar realizando recomendaciones poco útiles, por ejemplo, por ser demasiado similares \cite{ziegler}. Por tanto, en la mayoría de los casos resulta imposible establecer una medida sistemática del desempeño de un sistema y se hace necesario consultar directamente a los usuarios del sistema.