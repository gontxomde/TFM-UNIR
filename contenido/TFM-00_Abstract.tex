\begin{abstract}[Resumen]
En este proyecto analizamos la necesidad de utilizar sistemas de recomendación en el entorno tecnológico en el que vivimos. El incremento en el número de productos disponibles para los consumidores hace que sea prácticamente obligatorio para cualquier empresa que quiera ser competitiva disponer de uno de estos sistemas. Revisamos los diferentes tipos de sistemas de recomendación y sus implementaciones y creamos un sistema basado en contenido utilizando Python y Procesamiento del Lenguaje Natural para recomendar películas. Para realizar las recomendaciones se usarán las palabras clave de las películas y más metadatos disponibles. Se desarrollará una solución \textit{end to end}, no desarrollando únicamente el sistema sino también un entorno en el que podría ser utilizado. El resultado final consiste en paquete de Python que implementa una API integrada en un chatbot de Telegram que recomienda películas basándose en una aportada por el usuario.
\par\vspace{0.25cm}
\centering\textbf{Palabras clave: } \keywordsESv \par
\end{abstract}
\pagebreak
%Ingles
\begin{abstract}
In this project we discuss the need for recommendation systems in the technological enviroment we currently live on. The increase of the number of products available for the customers makes having a recommendation system a must for any online company looking for success. We review the different kinds of recommendation systems available and implement a content based one using Python and NLP to recommend films. An end to end solution will be developed, not only developing the system but also the enviroment in which it could be used. The final result consists on a Python  package that implements an API integrated in a Telegram chatbot. This chatbot recommends films based on one given by the user.\par
\par\vspace{0.25cm}
\centering\textbf{Keywords: } \keywordsv \par
\end{abstract}
