\chapter{Conclusiones y trabajo futuro}\label{chap:resultados}

En este trabajo hemos estudiado en profundidad los sistemas de recomendación. En primer lugar, se ha realizado una explicación de la necesidad actual de disponer de sistemas de recomendación de contenido, debido al cambio en los hábitos de consumo de los usuario y el crecimiento de los contenidos disponibles. Del mismo modo, se ha realizado una revisión teórica de los diferentes tipos de sistemas de recomendación utilizados en la actualidad (basados en contenido y colaborativos), explicando sus ventajas e inconvenientes, así como los casos de uso recomendados para cada uno de ellos.\\

Una vez explicada la necesidad de disponer de sistemas de recomendación, se ha obtenido un conjunto de datos de películas y series de televisión. Con este dataset se ha mostrado el flujo completo de preprocesado de los datos, explicándose las fases de importación, limpieza, transformación y completado de datos faltantes. Se ha podido ver cómo el conjunto de datos del que se disponía estaba en unas condiciones bastante buenas, ya que lo habitual es tener que realizar una fase de preprocesamiento de los datos significativamente más amplia.\\

Una vez se han tenido los datos en el formato correcto, se ha creado un sistema de recomendación basado en contenido, utilizando diferentes técnicas de NLP y de lógica difusa para obtener las palabras clave de cada película y transformarlas para evitar palabras diferentes que en el fondo tienen el mismo significado. El sistema de recomendación se ha encapsulado en un paquete de Python disponible de forma pública y con una documentación suficiente para poder entenderse su funcionamiento completamente.\\

Por último, y con el objetivo de realizar un proceso más end-to-end de la creación del sistema, se ha creado una API capaz de servir recomendaciones usando el sistema creado, a través de internet. La API se ha integrado con un chatbot de Telegram, posibilitando el uso del sistema por cualquier usuario y de forma gratuita.\\

Un proyecto de análisis de datos, por lo general, requeriría de una amplia discusión de los resultados obtenidos. Sin embargo, debido a la naturaleza del problema resuelto, no ha sido posible en este caso. Únicamente pueden evaluarse las recomendaciones de forma subjetiva, ya que volver a medir la similitud entre una película y las recomendadas sería repetir un proceso ya realizado en el flujo de creación del recomendador. En general, las recomendaciones de películas son lo suficientemente buenas, según se ha consultado con algunos usuarios. Una forma de evaluar los resultados obtenidos sería implementar el sistema en una Web y ver si los usuarios hacen más uso de las recomendaciones de este sistema que el de otros, es decir, lo que normalmente se llama un test A/B. Puede concluirse, por tanto, que los resultados son buenos, obteniendo unos tiempos de respuesta del chatbot relativamente buenos y unas recomendaciones valoradas como significativas por algunos usuarios preguntados.\\

Como trabajo futuro, creemos que sería útil explorar la vía de dar diferente importancia a cada una de las keywords, utilizando una implementación de la técnica tf-idf. Además, el tiempo necesario para generar una recomendación es suficiente para un chatbot usado por unos pocos usuarios, pero bastante alto como para un sistema productivo que fuera usado, por ejemplo, en una web como IMDB. Por tanto, de cara a una implementación a mayor escala, podría ser útil generar a priori las recomendaciones para todas las películas (unas 5000) y guardarlo en una tabla estilo Solr, de forma que se cree una API que únicamente enlace el título con un id y se busque ese id en Solr, sin generar las recomendaciones en el momento.