\chapter{Introducción}\label{chap:introduccion}
%Motivation

En los últimos años, con el crecimiento de plataformas online como Amazon, Netflix, YouTube y muchas otras, los contenidos disponibles para los usuarios han crecido de forma exponencial, habiéndose triplicado el número de películas lanzadas desde el año 2000 \cite{watson_2020} y multiplicado por 100 el número de horas subidas a YouTube cada minuto \cite{clement_2019}. \\

Hace no mucho tiempo si uno tenía que comprar un pequeño electrodoméstico para la cocina iba a una tienda especializada en ello. Si quería ver una película iba aun videoclub donde se le podían recomendar películas y alquilarlas, etc. Sin embargo, con el auge de Internet han aparecido plataformas que integran muchos de estos servicios. Amazon es un terrible centro comercial en el que podemos encontrar desde alimentación hasta arte, pasando por tecnología, libros... e incluso ver películas en su plataforma. YouTube, por su parte, es una plataforma en la que los propios creadores de contenido suben de forma diaria $26$ millones de horas de vídeo al día y $1000$ millones de horas son visualizadas cada día por sus $2000$ millones de usuarios \cite{YoutubeStats}. Queda a la vista el hecho de que en los últimos años ha tenido lugar un cambio en el orden de magnitud de las plataformas de venta, pasando de pequeñas tiendas o cadenas nacionales a plataformas internacionales.\\

Uno de los grandes retos que estas plataformas han tenido que abordar es mantener a sus usuarios en ellas. Pongamos por ejemplo una persona en un ordenador con los miles de millones de vídeos que hay en YouTube, o delante de todos los contenidos que hay disponibles para visualizar en Netflix (1770 títulos en España \cite{Lovely2019}). El elegir qué contenido consumir puede ser una tarea de tal magnitud, que consiga ahuyentar al usuario. Por tanto, parece que ya no es suficiente con tener en la plataforma (tienda, antes; web, ahora) contenidos atractivos para el usuario, ni siquiera es ya suficiente con conseguir los usuarios acceder a la plataforma en cuestión dispuestos a usarla, sino que es vital que sea la plataforma quien les ayude a filtrar el contenido que puede ser de su agrado.\\

Por su parte, los hábitos de consumo de los usuarios también han evolucionado mucho en los últimos años. Desde escuchar la opinión de un producto de uno o varios vendedores y de su círculo más cercano hasta contar en la actualidad con grandes redes de encuentro entre usuarios donde diferentes productos son valorados y analizados. En un ejemplo concreto, una persona que quiera comprar un equipo de música para su hogar, hace unos años hubiera ido a una tienda cercana a su casa, hubiera preguntado al vendedor y, en base a la opinión del mismo y su presupuesto, hubiera realizado una elección. En la actualidad, podría buscar equipos de música en Amazon, que le serían ordenados según opiniones de otros usuarios, hubiera podido leer cientos de opiniones personales de compradores, visto análisis de expertos en YouTube, que analizarían de forma más objetiva las ventajas e inconvenientes de cada producto.\\

Se puede decir que con la llegada de estas plataformas ha llegado también la necesidad de ser capaz de orientar al consumidor dentro de las mismas, ya que de no hacerlo existe el riesgo de que el usuario se vea abrumado y no sea capaz de utilizarla. Tampoco puede decirse que el beneficio de ser capaz de realizar una recomendación al usuario sea sólo para él. La forma que tiene YouTube de recomendarte vídeos puede hacer que lo que iba a ser una visita de $10$ minutos se transforme en media tarde viendo vídeos y, en consecuencia, anuncios.\\

Nuestras intenciones y las de las plataformas online, no tienen por qué estar alineadas. Las plataformas actuales tienen motores de recomendación potentes, capaces de mantener a sus usuarios en la plataforma. Sin embargo, muchos los usuarios lo que desean es tratar de ver el contenido que mas les pueda gustar, independientemente de la plataforma en la que se encuentre. Las compañías usan un acercamiento que tiene en cuenta primero la plataforma, ya que recomiendan contenido que ellos tienen disponible. Mientras tanto, muchos usuarios pueden preferir un acercamiento que dé prioridad al contenido, sin importar la plataforma.\\

Ha quedado patente que en la actualidad se hace necesario disponer de sistemas que en base al contenido disponible y la opinión de los distintos usuarios sean capaces de recomendar productos de la plataforma \cite{Konstan2004}. La intención del proyecto es desarrollar un motor de recomendación que, sin disponer de la misma cantidad de datos que plataformas como Netflix, sea capaz de obtener unos resultados de satisfacción comparables, pero sin tener en cuenta la plataforma de visionado del contenido.


%%%           %%%
%  Objectives   %
%%%           %%%
\section{Objetivos del proyecto}\label{sec:objetivos}

Como se ha visto anteriormente, es muy beneficioso para las empresas que ofrecen productos online ser capaces de recomendar a sus usuarios otros productos a partir de los ya consumidos. Esto tiene, dos ventajas fundamentales:

\begin{itemize}
    \item La plataforma es capaz de orientar al usuario, de forma que se evita una posible fuga de usuarios a otras plataformas ante la ingente cantidad de productos a consumir.
    \item Una vez el usuario ha consumido un producto pueden serle recomendados otros relacionados, incrementando el número de productos consumidos por el usuario.
\end{itemize}

En este trabajo utilizaremos un conjunto de datos compuesto por información sobre películas, con la finalidad de construir un sistema de recomendación de películas que, partiendo de una película propuesta por el usuario, pueda recomendarle otras similares. La diferencia con el motor de búsqueda de plataformas como Netflix, es que estará disponible fuera de estas plataformas, en una app tan común como Telegram, y que recomendará películas de todo tipo, no solo las que estén en una determinada plataforma. ¿Es posible crear un sistema de recomendación usando recomendación basada en contenido? ¿Es Telegram una buena vía para realizar esta recomendación? ¿Prefiere el usuario recomendaciones que no tengan en cuenta la plataforma, pudiendo ver contenido más variado? Estas son algunas de las preguntas que trataremos de resolver en este proyecto. La forma en la que tratremos de mejorar los sistemas de plataformas como Netflix o Prime Video será, por tanto, en términos de riqueza del contenido y de posibilidad de recibir recomendaciones en cualquier situación, no solo dentro de estas plataformas.\\

El conjunto de datos utilizado posee metadatos sobre las películas, como palabras clave, presupuesto, puntuación de los usuarios... Información relevante y suficiente para poder implementar este sistema. Como puede deducirse, es necesario un soporte para este sistema que permita al usuario final poderlo utilizar. Lo habitual sería integrarlo con un trabajo de User Experience (UX) y que quedara, por tanto, integrado en la plataforma. En este caso no disponemos de una plataforma en la que integrar el sistema, por lo que se creará una API (que contendrá el sistema de recomendación) que podrá ser llamada por un bot de Telegram, de forma que recomiende películas a un usuario a partir de otra.\\

\begin{figure}[h]
    \centering
    \captionsetup{width=10cm}
    \includegraphics[width=12cm]{contenido/imagenes/initial.png}
    \caption{Proceso de tratado de la información para la creación del sistema de recomendación.}
    \label{fig:process}
\end{figure}

El proceso que se seguirá para la creación del sistema de recomendación se muestra en la \autoref{fig:process} y consistirá de los siguientes pasos:
\begin{enumerate}
    \item Adquisición de los datos y limpieza de los mismos.
    %\item Análisis preliminar avanzado de los datos.
    \item Creación del sistema de recomendación.
    \item Creación de la API e integración en el bot de Telegram.
\end{enumerate}


\section{Estado del arte}\label{sec:soa}

En los últimos años ha habido un crecimiento en el interés de los sistemas de recomendación \cite{Abdomavicius}, desde la aparición de los primeros artículos al respecto a mediados de los años 90 \cite{resnick}. Por ejemplo, en \cite{nageswara} puede encontrarse un inventario de sistemas de recomendación existentes que han sido desarrollados tanto en un ámbito académico como en la industria.\\

En la creación de sistemas de recomendación hay dos grandes corrientes, que son las explicadas en el \autoref{chap:recom}. Existen motores de recomendación que se basan en atributos del contenido y sistemas de recomendación que se basan en opiniones e interacciones entre usuarios (filtrado colaborativo). Los últimos, normalmente requieren de una gran cantidad de datos, por lo que es habitual que grandes compañías como Amazon \cite{Amazon} o Google \cite{Google} las utilicen habitualmente. Además de requerir grandes cantidades de datos, este tipo de sistemas tienen el problema de que los elementos nuevos son difícilmente recomendables al carecer de opiniones. Los sistemas basados en contenido no tienen este problema, ya que la recomendación se efectúa, principalmente, de acuerdo con los metadatos de cada elemento. De esta forma, un elemento, por nuevo que sea, será recomendado siempre y cuando se hayan introducido los metadatos. Como nota adicional, hay que tener en cuenta que normalmente no se explica el funcionamiento de los mejores sistemas de recomendación, ya que hay compañías para las que una gran parte del negocio depende de estos sistemas.\\

En el filtrado colaborativo, el sistema recoge como entrada un conjunto de puntuaciones de los usuarios sobre una serie de items. Los usuarios pueden compararse comparando en puntuaciones compartidas o similares. Los items pueden compararse de similar manera con usuarios con votaciones similares. La puntuación hipotética de un usuario sobre un item puede predecirse usando información de usuarios de su vecindad y películas de la vecindad de la película a predecir. Dentro de los sistemas que usan filtrado colaborativo, pueden distinguirse: \textbf{filtros basados en usuarios} \cite{resnick}, donde la predicción se consigue mediante las puntuaciones de ese item dadas por usuarios similares. En este caso, se hace necesario definir una medida de similaridad entre usuarios. \textbf{Filtros basados en items}, en los que la predicción se usa buscando items similares \cite{Amazon}, \cite{sarwar}, \cite{karypis}. \textbf{Filtros basados en modelos}, los modelos anteriormente mencionados tienen una complejidad computacional cuadrática \cite{Candiller}, lo cual los hace inviables para conjuntos de datos muy grandes y con necesidades rápidas de respuesta. Estos problemas son lo que tratan de solucionar los filtros de este último tipo \cite{breese}. Estos modelos utilizan conceptos como el clustering para reducir la complejidad del problema, agregando los usuarios en clusters independientes que permitan reducir la complejidad \cite{oconnor}.\\

En los sistemas de recomendación basados en contenido, se realiza una recomendación al usuario usando como base una descripción del item y un perfil del usuario. Estos sistemas requieren de una forma de describir los items que pueden ser recomendados, una forma de modelar al usuario y una forma de comparar items. El perfil de usuario puede ser construido de forma implícita o explícita, a base de cuestioanrios. En nuestro caso, como se verá más adelante, se construirá de forma implícita a través de atributos del item dado por el usuario. En este tipo de filtros, como hemos dicho, es necesario poder comparar elementos textuales entre sí. Existen muchas formas de hacerlo, como por ejemplo mediante la distancia google \cite{cilibrasi} que infiere similaridades mediante la coocurrencia de estos elementos en páginas web. Sin embargo, esta medida tiene el problema de requerir de un servicio externo como es la API de GOogle para poder calcularla. En este proyecto, utilizaremos una distancia definida por la presencia de keywords compartidas (pasando antes por un preprocesado de las mismas), director y actores principales. Un problema que tienen en ocasiones los sistemas basados en contenido, es la sobreespecializacion \cite{zhang}. Es decir, que recomiendan elementos demasiado similares a los dados, lo que puede dar lugar a una falta de originalidad. Sin embargo, consideramos que no hay películas tan similares como para dar lugar a este problema.\\

Los sistemas de recomendación basados en filtrado colaborativo tienen un problema bastante importante, que se tratará en detalle más adelante en la memoria. Este problema es el llamado \textit{sparsity} en inglés. Esto es, que la mayoría de usuarios únicamente han votado una parte muy pequeña de las películas, por lo que la matriz que contiene las películas y las puntuaciones de cada usuario tiene una gran cantidad de ceros. En el Premio Netflix \cite{netflix} la solución que se propuso fue realizar una factorización de matrices, de forma que se condensa esta información e incluso se pueden obtener representaciones de las películas en un sistema de coordenadas equiparable a diferentes géneros. En \cite{ilhami2014film} se realiza una implementación de un sistema de recomendación basado en filtrado colaborativo, aunque como puede verse en el artículo, cuentan con una cantidad de datos mucho mayor. En la \autoref{tab:compareSystems} se realiza una comparación entre el sistema propuesto en este artículo y el propuesto en este proyecto.

\begin{table}[H]
\centering
%\resizebox{0.8\textwidth}{!}{%
\begin{tabular}{p{0.3\linewidth}p{0.3\linewidth}p{0.3\linewidth}}
\hilne \textbf{Aspecto} & \textbf{Sistema de este TFE} & \textbf{Sistema propuesto en \cite{ilhami2014film}} \\ \hline 
Recomendaciones dependientes del usuario & El sistema recomienda películas independientemente del usuario que las solicita.                                                                                & Cada usuario está modelado de forma diferente, por lo que las recomendaciones sí que son únicas                                                                                                    \\
Nuevos usuarios y elementos              & Al ser independiente del usuario, la aparicion de nuevos usuarios no es un problema. Los nuevos contenidos serán recomendados una vez se rellenen sus metadatos & Puede haber problemas al tener usuarios que no hayan valorado ninguna película o películas que no hayan sido recomendadas. Es el problema denominado \textit{cold start}                           \\
Posibilidad de indexación                & Podemos tener precalculadas las recomendaciones, lo cual supone un gran ahorro de tiempo a la hora de servir las recomendaciones a los usuarios                 & Tener las recomendaciones precalculadas es muy costoso, al depender del producto entre usuarios y películas.                                                                                       \\
Agnosticismo                             & El sistema recomienda las películas independientemente de la plataforma en la que se encuentran, al contrario que sistemas de Netflix o Amazon                  & El conjunto de películas utilizado no proviene de una plataforma en cocnreto. Sin embargo, al usar votos de usuarios, es complicado introducir una película que no esté en la plataforma de votos. \\
Implementación Real                      & El proyecto de este trabajo se ha desarrollado de una forma \textit{end to end}, de forma que se ha generado un sistema de recomendación usable                 & El sistema de recomendacion creado es de laboratorio, no hay posibilidad de usarlo                                                                                                                
\end{tabular}%
%}
\caption{Comparación entre el sistema de recomendación propuesto en \cite{ilhami2014film} y en esta memoria. El del artículo es un sistema basado en filtrado colaborativo y factorización de matrices, mientras que el de este trabajo es un filtro basado en contenido, con una heurística posterior de puntuación de las películas. Como puede verse, hay grandes diferencias entre ambos sistemas, destacando el de este proyecto por no necesitar tantos datos para funcionar, poder incluir nuevos elementos fácilmente y además estar implementado en un entorno real como el chatbot de Telegram.}
\label{tab:compareSystems}
\end{table}

La evaluación de los sistemas de recomendación es también un punto fundamental en este área. En la mayoría de los casos, la forma de realizar esta evaluación es comparando puntuaciones predecidas con las que se tienen originalmente, teniendo un conjunto de usuarios de entrenamiento y otro de test. Aunque esta información es relevante e interesante, en los sistemas de recomendación es fundamental evaluar la relevancia de las recomendaciones realizadas \cite{herlocker}, ya que aunque la predicción puede ser buena, puede estar realizando recomendaciones poco útiles, por ejemplo, por ser demasiado similares \cite{ziegler}. Por tanto, en la mayoría de los casos resulta imposible establecer una medida sistemática del desempeño de un sistema y se hace necesario consultar directamente a los usuarios del sistema.
%%%                  %%%
% Document structure   %
%%%                  %%%
\section{Estructura de la memoria}\label{sec:estructura}

El trabajo, como se ha descrito en los capítulos anteriores, tiene como objetivo la creación de un sistema de recomendación de películas, aunque se tratará de incluir en el mismo el flujo \textit{end to end} de tratamiento de la información. El trabajo se estructurará en diferentes capítulos, conteniendo cada uno de ellos una de las partes del flujo. Los capítulos que se incluirán en este trabajo serán los siguientes:

\begin{itemize}
    \item \autoref{chap:recom}: Sistemas de recomendación. Se realizará una explicación de la teoría que existe tras estos sistemas y los diferentes tipos que existen. Se creará el sistema de recomendación.
    \item \autoref{chap:adq}: Estructura de los datos a utilizar y preprocesado de los mismos. En este capítulo se realizará una presentación y explicación de los datos de los que se dispone, de forma que se entiendan los siguientes pasos seguidos.
    %\item \textit{Preliminary Data Analysis}. Caso de uso de análisis de datos para resolución de preguntas de negocio. %Al trabajar con un conjunto de datos siempre es importante realizar un análisis previo, de forma que pueda %entenderse la estructura de los mismos. También se verá la utilidad de disponer de un conjunto de datos %relativamente grande a la hora de responder a preguntas que pueden surgir en relación con la informacion contenida %en los mismos.
    
    \item \autoref{chap:api}: Integración del sistema en una API y creación de un chatbot. Es importante que nuestro sistema de recomendación sea explotable por los usuarios, de nada sirve tener un buen sistema de recomendación si no puede ser usado. Se explicará la creación de un bot de Telegram y cómo crear una API que pueda ser llamada por el bot.
    \item \autoref{chap:resultados}: Próximos pasos. Finalmente, se realizará un repaso de lo desarrollado en el trabajo y se propondrán puntos en los que mejorar el sistema que no hayan podido implementarse por algún motivo.
\end{itemize}

\section{Trabajo desarrollado}\label{sec:trabajodesarrollado}

En esta memoria, se trata de solventar el problema que se ha comentado en las secciones anteriores. Las plataformas de contenido disponen de sistemas de recomendación potentes, pero con una limitación de origen: solo recomiendan contenido de sus plataformas. Esto hace que si vemos una película concreta en una plataforma y esta nos recomienda otra, es posible que haya mejores recomendaciones que la plataforma en cuestión no esté realizando debido a que no dispone de ese contenido.\\ 

Tal y como se verá en el \autoref{chap:recom}, un tipo de motores de recomendación son los motores basados en contenido. Estos motores se basan en características intrínsecas de las películas y no tanto en las opiniones atómicas de los usuarios. En la solución propuesta en este trabajo, las palabras clave que describen una película son la parte central de la recomendación. Esta elección se ha realizado principalmente por los datos disponibles, ya que los temas de una película, su año de producción o la opinión media de los usuarios son datos relativamente abiertos, mientras que las opiniones concretas de cada usuario son algo propiedad de las plataformas en las que se realizan estas valoraciones, por lo que no tenemos acceso a estos datos.\\ 

Para realizar la recomendación y una vez se ha limpiado el dataset, las películas se distribuyen en un espacio vectorial de keywords y se seleccionan en base a una distancia las más próximas a la película dada. Una vez se tienen las películas más cercanas a la película dada, se establece una heurística que puntúa cada una de estas películas en función de su puntuación, su año de producción y el ratio entre votos e ingresos. Esta heurística trata de, una vez se han seleccionado las películas candidatas a ser recomendadas, tratar de escoger de entre ellas las que probablemente prefiera el usuario. Por ejemplo, creemos que si tras ver una película de Harry Potter se nos recomienda una de magia similar pero del año 1950 es probable que el usuario sea reacio a escogerla.\\

Una de las principales ventajas de nuestro motor de recomendación, es la forma en la que se ha implementado. Al ser un motor agnóstico en términos de plataforma (ya que simplemente tiene datos de las películas y su puntuación en IMDB), necesitábamos un sitio desde el que pudiera ser explotado. De entre todos los posibles, hemos elegido implementarlo en un bot de Telegram, ya que Telegram es una plataforma con una gran expansión y además multiplataforma. Consideramos que la mayoría de usuarios tendrán a mano un dispositivo móvil cuando estén viendo películas, por lo que en cualquier momento pueden escribir un título al bot y éste les enviará las recomendaciones obtenidas.\\

Además, nuestro motor (y en general los motores basados en contenido) tiene una ventaja principal, y es que puede ser actualizado fácilmente con la llegada de nuevas películas a la cartelera. Dado que la información principal que utilizamos son las palabras clave, no necesitamos que un gran número de usuarios haya visto la película en cuestión para que tengamos la información suficiente para recomendar esa nueva película. Esto hace que sea un modelo especialmente beneficioso para que los usuarios puedan descubrir películas menos conocidas que puedan ser de su agrado.\\

Uno de los problemas que suelen tener los motores de recomendación basados en contenido es que en muchas ocasiones recomiendan elementos demasiado similares. Esto se entiende mejor con un ejemplo: imaginemos que nuestra última compra en Amazon es un teclado de ordenador. Si Amazon nos recomendase los productos más parecidos, probablemente nos recomendase otros teclados. Normalmente, cuando hemos comprado un teclado lo último que necesitamos es otro. Las recomendaciones serían, en este caso, tan similares que no serían de utilidad. Sin embargo, consideramos que como nuestro sistema sirve para recomendar películas, este problema se ve mitigado por el mero hecho de que no hay películas tan similares entre sí como productos en Amazon, por ejemplo.


    
