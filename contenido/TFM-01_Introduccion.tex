\chapter{Introducción}\label{chap:introduccion}
%Motivation

En los últimos años, con el crecimiento de plataformas online como Amazon, Netflix, YouTube y muchas otras, los contenidos disponibles para los usuarios han crecido de forma inimaginable. \\

Hace no mucho tiempo si uno tenía que comprar un pequeño electrodoméstico para la cocina iba a una tienda especializada en ello, si quería ver una película iba aun videoclub donde se le podían recomendar películas y alquilarlas, etc. Sin embargo, con el auge de Internet han aparecido plataformas que integran muchos de estos servicios. Amazon es un terrible centro comercial en el que podemos encontrar desde alimentación hasta arte, pasando por tecnología, libros... e incluso ver películas en su plataforma. YouTube, por su parte, es una plataforma en la que los propios creadores de contenido suben de forma diaria $26$ millones de horas de vídeo al día y $1000$ millones de horas son visualizadas cada día por sus $2000$ millones de usuarios \cite{YoutubeStats}. Queda a la vista el hecho de que en los últimos años ha tenido lugar un cambio en el orden de magnitud de las plataformas de venta, pasando de pequeñas tiendas o cadenas nacionales a plataformas internacionales.\\

Uno de los grandes retos de estas plataformas es mantener a sus usuarios en ellas. Imaginemos por ejemplo una persona en un ordenador con los miles de millones de vídeos que hay en YouTube, o delante de todos los contenidos que hay disponibles para visualizar en Netflix. El elegir qué contenido consumir puede ser una tarea de tal magnitud, que consiga ahuyentar al usuario. Por tanto, parece que ya no es suficiente con tener en la plataforma (tienda, antes; web, ahora) contenidos atractivos para el usuario, ni siquiera es ya suficiente con conseguir los usuarios acceder a la plataforma en cuestión dispuestos a usarla, sino que es vital que sea la plataforma quien les ayude a filtrar el contenido que puede ser de su agrado.\\



Por su parte, los hábitos de consumo de los usuarios también han evolucionado mucho en los últimos años. Desde escuchar la opinión de un producto de uno o varios vendedores y de su círculo más cercano hasta contar en la actualidad con grandes redes de encuentro entre usuarios donde diferentes productos son valorados y analizados. En un ejemplo concreto, una persona que quiera comprar un equipo de música para su hogar, hace unos años hubiera ido a una tienda cercana a su casa, hubiera preguntado al vendedor y, en base a la opinion del mismo y su presupuesto, hubiera realizado una elección. En la actualidad, podría buscar equipos de música en Amazon, que le serían ordenados según opiniones de otros usuarios, hubiera podido leer cientos de opiniones personales de compradores, visto análisis de expertos en YouTube, que analizarían de forma más objetiva las ventajas e inconvenientes de cada producto.\\

Se puede decir que con la llegada de estas plataformas ha llegado también la necesidad de ser capaz de orientar al consumidor dentro de las mismas, ya que de no hacerlo existe el riesgo de que el usuario se vea abrumado y no sea capaz de utilizarla. Tampoco puede decirse que el beneficio de ser capaz de realizar una recomendación al usuario sea sólo para él. La forma que tiene YouTube de recomendarte vídeos puede hacer que lo que iba a ser una visita de $10$ minutos se transforme en media tarde viendo vídeos y, en consecuencia, anuncios.\\

Ha quedado patente que en la actualidad se hace necesario disponer de sistemas que en base al contenido disponible y la opinión de los distintos usuarios sean capaces de recomendar productos de la plataforma.


%%%           %%%
%  Objectives   %
%%%           %%%
\section{Objetivos}\label{sec:objetivos}

Como se ha visto anteriormente, es muy beneficioso para las empresas que ofrecen productos online ser capaces de recomendar a sus usuarios otros productos a partir de los ya consumidos. Esto tiene, dos ventajas fundamentales:

\begin{itemize}
    \item La plataforma es capaz de orientar al usuario, de forma que se evita una posible fuga de usuarios ante la ingente cantidad de productos a consumir.
    
    \item Una vez el usuario ha consumido un producto pueden serle recomendados otros relacionados, incrementando el número de productos consumidos por el usuario.
\end{itemize}

En este trabajo utilizaremos un conjunto de datos compuesto por información sobre películas, con la finalidad de construir un sistema de recomendación de películas que, partiendo de una película propuesta por el usuario, pueda recomendarle otras similares.\\

El conjunto de datos posee metadatos sobre las películas, como palabras clave, presupuesto, puntuación de los usuarios... Información relevante y suficiente para poder implementar este sistema. Como puede deducirse, es necesario un soporte para este sistema que permita al usuario final poderlo utilizar. Lo habitual sería integrarlo con un trabajo de User Experience (UI) y que quedara, por tanto, integrado en la plataforma. En este caso no disponemos de una plataforma en la que integrar el sistema, por lo que se creará una API (que contendrá el sistema de recomendación) que podrá ser llamada por un bot de Telegram, de forma que recomiende películas a un usuario a partir de otra.\\

\begin{figure}[h]
    \centering
    \captionsetup{width=10cm}
    \includegraphics[width=10cm]{contenido/imagenes/DDSD.pdf}
    \caption{Proceso de tratado de la información para la creación del sistema de recomendación.}
    \label{fig:process}
\end{figure}

El proceso que se seguirá para la creación del sistema de recomendación se muestra en la Figura \ref{fig:process} y consistirá de los siguientes pasos:
\begin{enumerate}
    \item Adquisición de los datos y limpieza de los mismos.
    \item Análisis preliminar avanzado de los datos.
    \item Creación del sistema de recomendación.
    \item Creación de la API e integración en el bot de Telegram.
\end{enumerate}


%%%                  %%%
% Document structure   %
%%%                  %%%
\section{Estructura del documento}\label{sec:estructura}

El trabajo, como se ha descrito en los capítulos anteriores, tiene como objetivo la creación de un sistema de recomendación de películas, aunque se tratará de incluir en el mismo el flujo \textit{end to end} de tratamiento de la información. El trabajo se estructurará en diferentes capítulos, contieniendo cada uno de ellos una de las partes del flujo. Los capítulos que se incluirán en este trabajo serán los siguientes:

\begin{itemize}
    \item Estructura de los datos a utilizar y preprocesado de los mismos. En este capítulo se realizará una presentación y explicación de los datos de los que se dispone, de forma que se entiendan los siguientes pasos seguidos.
    \item \textit{Preliminary Data Analysis}. Caso de uso de análisis de datos para resolución de preguntas de negocio. Al trabajar con un conjunto de datos siempre es importante realizar un análisis previo, de forma que pueda entenderse la estructura de los mismos. También se verá la utilidad de disponer de un conjunto de datos relativamente grande a la hora de responder a preguntas que pueden surgir en relación con la informacion contenida en los mismos.
    \item Sistemas de recomendación. Se realizará una explicación de la teoría que existe tras estos sistemas y los diferentes tipos que existen. Se creará el sistema de recomendación.
    \item Integración del sistema en una API y creación de un chatbot. Es importante que nuestro sistema de recomendación sea explotable por los usuarios, de nada sirve tener un buen sistema de recomendación si no puede ser usado. Se explicará la creación de un bot de Telegram y cómo crear una API que pueda ser llamada por el bot.
    \item Próximos pasos. Finalmente, se realizará un repaso de lo desarrollado en el trabajo y se propondrán puntos en los que mejorar el sistema que no hayan podido implementarse por algún motivo.
\end{itemize}

    
