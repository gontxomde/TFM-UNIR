\chapter{Objetivos y Metodología}\label{chap:objetivos}

Este trabajo tiene como objetivo presentar un caso de uso de las herramientas estudiadas en el máster. Como se ha descrito en capítulos anteriores, el caso de uso elegido es partir de fuentes de datos relacionadas con películas y desarrollar un \textbf{sistema de recomendación} que sea capaz de dar al usuario películas similares a las dadas por él.\\

Para ello se realizará un \textbf{análisis de los datos} y una presentación teórica de los diferentes modelos que existen para crear un sistema de recomendación. Finalmente, se escogerá una técnica concreta en función de sus ventajas y los datos disponibles y se implementará. Para que pueda ser usado en un entorno más productivo y fuera de un ordenador particular, se integrará en el flujo un chatbot que pueda recibir la información del usuario, procesarla, y darle recomendaciones haciendo llamadas a una \textbf{API} que será creada.



%--- Genereal Objectives ---%
\section{Objetivos Generales}\label{sec:objgenerales}

Crear un sistema de recomendación de películas agnóstico a la plataforma en la que el contenido está disponible y utilizable en cualquier entorno. Con este fin, se usarán los datos disponibles en: \url{https://tinyurl.com/ypfdw8w}\\

En el mundo actual los usuarios disponen de una gran cantidad de recursos a consumir cuando entran a una plataforma como YouTube o Netflix. Por tanto, se hace necesario que las plataformas sean capaces de entender sus \textbf{gustos y preferencias} para que el cliente consuma la mayor cantidad de contenido posible. En este trabajo se explora la vía de creación de un sistema de recomendación basado en contenido que, a partir de modelos de lenguaje sea capaz de entender el contenido de cada una de las películas y a partir de una preferencia dada por el usuario, le pueda recomendar otras películas similares para ver. Además, destruirá la barrera de la plataforma que imponen los sistemas embebidos dentro de estas. El sistema desarrollado recomendará sin tener en cuenta dónde está disponible el contenido, con el fin de ampliar los horizontes del usuario.\\

Como se verá más adelante, en el \autoref{chap:recom}, existen diferentes tipos de sistemas de recomendación. En este trabajo nos centraremos en sistemas basados en contenido, dado que tienen ventajas como su capacidad para recomendar productos nuevos y la no necesidad de información del usuario. Además, la información sobre las películas es más sencilla de obtener en la red.\\

Explicado esto, este trabajo se engloba dentro de los trabajos de Tipo 2 (desarrollo de software). Se realizará un desarrollo de una solución \textit{end to end} al problema de la recomendación de contenidos a usuarios. Desde la recopilación de datos y metadatos sobre películas, hasta la creación de la interfaz a través de la que el usuario final podrá hacer solicitudes al sistema creado y obtener recomendaciones.\\

Además, se realizará una revisión de los diferentes sistemas de recomendación existentes en la actualidad y su aparición en los últimos años.\\

%Finalmente, creemos necesario realizar un análisis de los datos que se utilizaran, previo a la creación del sistema de recomendación. En el %campo de la Ciencia de Datos se considera una buena práctica la realización de un análisis de los datos de forma previa a la creación del %sistema de recomendación en sí. Además de este análisis preliminar, mostraremos cómo el disponer de un conjunto lo suficientemente extenso de %datos puede servir para responder preguntas sobre los usuarios, que pueden ayudar al desarrollo del negocio.\\


%--- Specific Objectives ---%
\section{Objetivos Específicos}\label{sec:objespecificos}

Una vez descritos los objetivos generales del trabajo que se desarrollará, se hace necesario desgranarlos en una serie de objetivos más específicos que puedan ser más fácilmente analizados y desarrollados y para los que pueda establecerse un grado de consecución.

\begin{itemize}
    \item Conocer las fuentes de datos disponibles, el formato de los datos de películas y actores y sus posibilidades de explotación.
    \item Describir los datos disponibles y analizar los datos para sacar conclusiones sobre los mismos.
    %\item \textbf{IDENTIFICAR} tendencias a partir de los datos y \textbf{RESPONDER} a partir de ellos a preguntas que puedan surgir.
    \item Desarrollar el sistema de recomendación de películas que permita a los usuarios obtener recomendaciones a partir de una película dada..
    \item Integrar la solución desarrollada en un entorno API-bot que pueda ser usada por usuarios finales.
    \item Verificar la usabilidad de la solución y evaluar el rendimiento de la misma.
    \item identificar las flaquezas del sistema y los puntos de mejora para proponer próximos pasos.
\end{itemize}

%--- Work Methodology ---%
\section{Metodología de Trabajo}\label{sec:metodología}

En este trabajo, se ha utilizado una metodología definida de forma inicial, aunque posteriormente se han tenido que ir realizando adaptaciones de la misma a medida que se desarrollaba el mismo.\\

En primer lugar, hay se definió de forma concreta el problema que se quería resolver y se realizó una investigación bibliográfica con la finalidad de conocer los diferentes sistemas de recomendación que existen, de forma que pudiéramos definir correctamente el tipo de sistema que se quería utilizar y los datos necesarios.\\

Partiendo del problema descrito en la introducción, se buscó un conjunto de datos que hiciera posible dar una solución al problema. El conjunto de datos disponible está disponible en \url{https://github.com/harshitcodes/tmdb_mo}.\\

Una vez encontrados unos datos con los que trabajar, se definieron los pasos a seguir para la implementación del proyecto y la resolución del problema:
\begin{enumerate}
    \item Limpieza de los datos: En un proyecto de análisis de datos real, es una fase imprescindible. En esta fase se revisan los datos originales, se borran los datos incorrectos y se tratan los valores incorrectos o faltantes. Las herramientas a utilizar en esta etapa serán principalmente Python y la librería Pandas.
    \item Preprocesamiento de los datos: Una vez se hayan limpiado los datos, será necesario realizar un preprocesamiento de los mismos para que sirvan de entrada al sistema de recomendación, fase en la que se usará Python y librerías de procesamiento de lenguaje natural si fuera necesario.
    \item Construcción del sistema de recomendación: Tras la investigación bibliográfica y el preprocesamiento de los datos, se construirá el sistema de recomendación en sí. El objetivo es poder dar 3 recomendaciones de películas dada una de entrada, que pueda tener faltas de ortografía, de forma que el sistema sea más flexible.
    \item Integración del sistema en una API: Desde el principio, la intención ha sido crear un sistema usable y desde una perspectiva \textit{end to end}. En muchas ocasiones este tipo de trabajos se quedan en una mera investigación bibliográfica y teórica. Sin embargo, la intención es que el sistema sea realmente usable por todo el mundo que lo desee. En esta fase se valorarán las diferentes alternativas para proveer de una interfaz al sistema creado, generando un servicio web desde el que puedan realizarse recomendaciones.
\end{enumerate}