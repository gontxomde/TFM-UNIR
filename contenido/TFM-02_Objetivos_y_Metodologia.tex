\chapter{Objetivos y Metodología}\label{chap:objetivos}
\textbf{Objetivos concretos y metodología de trabajo}
Este bloque es el puente entre el estudio del dominio y la contribución a realizar. Según el tipo concreto de trabajo, el bloque se puede organizar de distintas formas, pero los siguientes elementos deberían estar presentes con mayor o menor detalle.
Objetivo general

%--- Genereal Objectives ---%
\section{Objetivos Generales}\label{sec:objgenerales}
Los trabajos aplicados se centran en conseguir un impacto concreto, demostrando la
efectividad de una tecnología, proponiendo una nueva metodología o aportando nuevas
herramientas tecnológicas. El objetivo por tanto no debe ser sin más ``crear una
herramienta'' o ``desarrollar una metodología'', sino que debe centrarse en conseguir un efecto observable.\par
Este objetivo podría dar lugar a un trabajo de tipo 3 (desarrollo de metodologías y
frameworks) que plantease una nueva metodología de captura de datos basada en la
aplicación de APIs públicas como TinCan, acompañada de guías de desarrollo,
ejemplos de uso y guías de buenas prácticas para desarrolladores; o de tipo 2
(desarrollo de software) en la que se genere un sistema de recopilación de eventos para entornos cloud basado en TinCan.


%--- Specific Objectives ---%
\section{Objetivos Específicos}\label{sec:objespecificos}
Independientemente del tipo de trabajo, la hipótesis o el objetivo general típicamente se dividirán en un conjunto de objetivos más específicos analizables por separado. Suelen ser explicaciones de los diferentes pasos a seguir en la consecución del objetivo general.
Con los objetivos, has de concretar qué pretendes conseguir. Se formulan con un verbo en infinitivo más el contenido del objeto de estudio. Se pueden utilizar fórmulas verbales, como las siguientes:\par
{\centering
ANALIZAR – CALCULAR – CLASIFICAR – COMPARAR – CONOCER -
CUANTIFICAR – DESARROLLAR - DESCRIBIR – DESCUBRIR - DETERMINAR –
ESTABLECER – EVALUAR - EXPLORAR -IDENTIFICAR –INDAGAR - MEDIR –
PROPOPONER - SINTETIZAR – VERIFICAR
}\par

En un trabajo como el anterior (tipo 2), incluiría tareas tales como ``Identificar las posibilidades de captura de datos del escenario'', ``Diseñar e implementar la pasarela de datos que permita enviar los registros a través de TinCan'', ``Desarrollar un manual de uso para poder añadir nuevas plataformas al entorno de captura de datos'', ``Proponer el modelo de datos bajo el que se almacenarán los datos registrados'' o ``Evaluar la herramienta implementada en un escenario real''.

%--- Work Methodology ---%
\section{Metodología de Trabajo}\label{sec:metodología}

De cara a alcanzar los objetivos específicos (y con ellos el objetivo general o la
validación/refutación de la hipótesis), será necesario realizar una serie de pasos. La metodología del trabajo debe describir qué pasos se van a dar, el porqué de cada paso, qué instrumentos se van a utilizar, cómo se van a analizar los resultados, etc.