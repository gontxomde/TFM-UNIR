\chapter{Evaluación del sistema}\label{chap:resultados}

En este trabajo hemos estudiado en profundidad los sistemas de recomendación. En primer lugar, se ha realizado una explicación de la necesidad actual de disponer de sistemas de recomendación de contenido, debido al cambio en los hábitos de consumo de los usuario y el crecimiento de los contenidos disponibles. Del mismo modo, se ha realizado una revisión teórica de los diferentes tipos de sistemas de recomendación utilizados en la actualidad (basados en contenido y colaborativos), explicando sus ventajas e inconvenientes, así como los casos de uso recomendados para cada uno de ellos.\\

Una vez explicada la necesidad de disponer de sistemas de recomendación, se ha obtenido un conjunto de datos de películas y series de televisión. Con este dataset se ha mostrado el flujo completo de preprocesado de los datos, explicándose las fases de importación, limpieza, transformación y completado de datos faltantes. Se ha podido ver cómo el conjunto de datos del que se disponía estaba en unas condiciones bastante buenas, ya que lo habitual es tener que realizar una fase de preprocesamiento de los datos significativamente más amplia.\\

Una vez se han tenido los datos en el formato correcto, se ha creado un sistema de recomendación basado en contenido, utilizando diferentes técnicas de NLP y de lógica difusa para obtener las palabras clave de cada película y transformarlas para evitar palabras diferentes que en el fondo tienen el mismo significado. El sistema de recomendación se ha encapsulado en un paquete de Python disponible de forma pública y con una documentación suficiente para poder entenderse su funcionamiento completamente.\\

Por último, y con el objetivo de realizar un proceso más end-to-end de la creación del sistema, se ha creado una API capaz de servir recomendaciones usando el sistema creado, a través de internet. La API se ha integrado con un chatbot de Telegram, posibilitando el uso del sistema por cualquier usuario y de forma gratuita.\\

Normalmente, en un proyecto de análisis de datos y de aprendizaje automático, se utilizan técnicas estadísticas de análisis de resultados. Sin embargo, como puede verse, en nuestro caso los resultados están sujetos a una evaluación subjetiva. Esto es, no existen unas predicciones que nuestro modelo deba ser capaz de reproducir, como ocurre muchas veces en Machine Learning, sino que se han utilizado técnicas de aprendizaje no supervisado y serán los usuarios del motor de recomendación los que tengan que evaluar los resultados en base a sus preferencias personales. Con la finalidad de evaluar el trabajo realizado y, en consecuencia, valorar el logro de los objetivos planteados en el \autoref{chap:objetivos}, se ha realizado una encuesta a 10 personas reales a las que se les ha descrito el proyecto y que han usado el servicio.\\

Es importante explicar con detalle la metodología seguida para la obtención de los resultados que nos permitirán realizar la evaluación del sistema. Debido a la forma de implementación del sistema, la encuesta se ha realizado a 10 personas de entre 20 y 35 años, ya que es un rango de edad en el que consideramos que un sistema como el desarrollado en este trabajo puede tener un mayor éxito. Los pasos para la evaluación del sistema son los siguientes:

\begin{enumerate}
    \item Se presenta a cada usuario, de forma individual, el sistema desarrollado, dándole una presentación y una motivación del mismo. Al usuario se le explican las diferencias entre los sistemas de la mayoría de plataformas y este, que es agnóstico y puede hacer recomendaciones independientemente de la plataforma en la que puedan consumirse las películas.
    \item Se le pide al usuario que a lo largo de una semana realice peticiones al servicio y que anote el número de peticiones que ha realizado.
    \item Al final de la semana, se le envía al usuario una encuesta para que evalúe el sistema. Esta encuesta es la que finalmente se nos remite para poder almacenar el resultado.
\end{enumerate}

Consideramos que aun haciendo inicialmente una explicación del sistema, las respuestas sobre su necesidad no se ven demasiado influidas por esta dado que la encuesta se responde al cabo de 7 días, evitando puntuaciones altas ``en caliente''. Las preguntas realizadas en la encuesta han sido las siguientes:

\begin{itemize}
    \item \textbf{Necesidad}: Valoración del 0 al 10 de la utilidad que ve el usuario a un sistema de recomendación de películas que es agnóstico en términos de la plataforma.
    \item \textbf{Número de usos}: Veces que el usuario ha usado el servicio.
    \item \textbf{Recomendaciones}: Valoración del 0 al 10 de las recomendaciones realizadas por el servicio.
    \item \textbf{Uso}: Valoración del 0 al 10 de la experiencia de usuario del servicio.
    \item \textbf{Repetición}: Cómo de probable es, del 0 al 10, que vuelvas a usar el servicio.
\end{itemize}

En la \autoref{tab:satisfaction} se muestran los resultados obtenidos en la encuesta. Con cada una de las columnas, podemos sacar una conclusión sobre cada una de las partes del proyecto. En primer lugar, puede verse que los usuarios valoran con una media de $6,8$ la necesidad de una plataforma que les permita obtener recomendaciones de contenido sin tener en cuenta la plataforma en la que se encuentren. Al final, a muchos usuarios les gusta realmente el cine y no lo ven como puro entretenimiento, por lo que era de esperar que una vez presentada la propuesta a los usuarios, les pareciera interesante. Como puede observarse en la Tabla, los usuarios han valorado con un $7,2$ las valoraciones realizadas por el sistema. Hay que tener en cuenta que el sistema únicamente realiza 5 recomendaciones (aunque podría aumentarse este valor), mientras que las plataformas normalmente recomiendan decenas de contenidos. Un punto clave del proyecto es la experiencia de usuario. En el desarrollo del proyecto consideramos que era más cómodo para el usuario tener una conversación de Telegram en la que poder obtener (y tener guardadas) las recomendaciones realizadas por el sistema. Además, el usar directamente un servicio multiplataforma como es Telegram, daba de inicio bastante facilidad de uso a los potenciales usuarios. Finalmente, los usuarios valoran con un $7,5$ sobre $10$ la probabilidad de que vuelvan a usar el servicio en el futuro.

\begin{table}[H]
\centering
\resizebox{0.8\textwidth}{!}{%
\begin{tabular}{cccccc}
 \hline \textbf{Usuario} & \textbf{Necesidad} & \textbf{Número de usos} & \textbf{Recomendaciones} & \textbf{Uso} & \textbf{Repetición} \\ \hline
1                                     & 6                                       & 5                                            & 8                                             & 7                                 & 8                                       \\
2                                     & 7                                       & 8                                            & 7                                             & 8                                 & 7                                       \\
3                                     & 6                                       & 6                                            & 6                                             & 9                                 & 5                                       \\
4                                     & 7                                       & 3                                            & 8                                             & 8                                 & 8                                       \\
5                                     & 7                                       & 5                                            & 9                                             & 8                                 & 6                                       \\
6                                     & 6                                       & 7                                            & 6                                             & 8                                 & 7                                       \\
7                                     & 6                                       & 8                                            & 7                                             & 7                                 & 9                                       \\
8                                     & 7                                       & 4                                            & 6                                             & 8                                 & 8                                       \\
9                                     & 7                                       & 4                                            & 8                                             & 9                                 & 7                                       \\
10                                    & 9                                       & 6                                            & 7                                             & 8                                 & 9                                      
\end{tabular}%
}
\caption{Encuesta realizada sobre una muestra de usuarios a los que les fue presentado el servicio. Necesidad muestra cómo de necesaria ven la plataforma los usuarios.}
\label{tab:satisfaction}
\end{table}

Cabe destacar, que algunos de los usuarios encuestados ya habían detectado este problema de los sistemas de recomendación de las plataformas de vídeo. En concreto, un usuario comentaba cómo le era difícil obtener recomendaciones de cine de películas de hace más de 50 años, ya que no era el principal objetivo de las plataformas. El año es uno de los factores determinantes en las recomendaciones realizadas por nuestro sistema, por lo que este usuario sí podría obtener recomendaciones de películas antiguas dada una que ya haya visto. En plataformas como Netflix, muchas veces nos invita a ver simplemente nuevas producciones multimillonarias que se recomiendan de forma extensiva, sin haber demasiada relación entre esa nueva producción y las anteriores que hemos visualizado. Nuestra plataforma carece de esos intereses ocultos, y su fin último es recomendar películas similares a los usuarios sin dar preferencia de forma apriorística a ninguna otra.\\

Para evaluar los resultados obtenidos, también hay que tener en cuenta la cantidad de datos de los que disponen plataformas como Netflix y el conjunto de datos con el que hemos trabajado en este TFE. Resulta evidente que se ha trabajado con una cantidad de datos bastante reducida, por lo que es de esperar que la calidad de las recomendaciones sea inferior. Sin embargo, el haber dotado al sistema de esta transversalidad interplataforma, hace que los usuarios valoren las recomendaciones y el servicio en su conjunto, ya que casi todos volverían a utilizar el servicio. Además, se ha conseguido una API que de unos tiempos de respuesta de alrededor de $1 s$. Este sistema podría escalarse usando una tecnología como Docker Swarm o Kubernetes, dado que las peticiones son absolutamente independientes.\\

\chapter{Conclusiones y trabajo Futuro}\label{chap:futuro}

El mundo de los recomendadores de contenido, es difícil de explorar, ya que muchas de las soluciones son la base de algunas empresas, por lo que no es sencillo acceder a los sistemas más avanzados. Sin embargo, analizando los objetivos específicos que se planteaban en la \autoref{sec:objespecificos} podemos comprobar el éxito o no del proyecto:
\begin{itemize}
    \item Se han conocido y analizado las fuentes de datos disponibles para el propósito que queríamos.
    \item Hemos descrito la información de los datos disponibles y hemos analizado su contenido más relevante en la construcción del sistema de recomendación de películas.
    \item Hemos desarrollado el sistema de recomendación de películas, usando técnicas de NLP y Python.
    \item Hemos integrado el sistema de recomendación creado en una API que puede ser usada de forma real por los usuarios finales. La forma seleccionada para explotar esta API ha sido la creación de un bot de Telegram.
    \item Hemos verificado y evaluado la usabilidad y utilidad del sistema creado, mediante un sondeo con usuarios reales del mismo.
\end{itemize}

El resultado final es un sistema de recomendación con una puntuación relativamente alta por parte de los usuarios encuestados que pone como prioridad la recomendación de la mejor película posible así como la protección de los datos del usuario.\\

Como trabajo futuro, creemos que sería útil explorar la vía de dar diferente importancia a cada una de las keywords, utilizando una implementación de la técnica tf-idf. Además, el tiempo necesario para generar una recomendación es suficiente para un chatbot usado por unos pocos usuarios, pero bastante alto como para un sistema productivo que fuera usado, por ejemplo, en una web como IMDB. Por tanto, de cara a una implementación a mayor escala, podría ser útil generar a priori las recomendaciones para todas las películas (unas 5000) y guardarlo en una base de datos estilo Solr, de forma que se cree una API que únicamente enlace el título con un id y se busque ese id en Solr, sin generar las recomendaciones en el momento. Esta posibilidad es una de las ventajas de los sistemas de recomendación basados en contenido, dado que la recomendación es independiente del usuario que la realiza.