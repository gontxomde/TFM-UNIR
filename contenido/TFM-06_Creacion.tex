\chapter{Creación del sistema de recomendación}\label{chap:creacion}


En los capítulos anteriores se ha realizado una introducción al trabajo desarrollado en este proyecto, cuyo objetivo es la creación de un sistema de recomendación de películas basado en contenido. En dicha introducción, se ha realizado una justificación de la importancia de estos sistemas para las plataformas que alojan contenido así como para los usuarios.

Además, en el \autoref{chap:adq} se ha realizado el proceso de limpieza y preprocesameinto de los datos que, como se ha dicho, es fundamental para lograr un buen desempeño del modelo en esta parte.\\

Recordar que en este momento disponemos de un conjunto de datos de aproximadamente 5000 películas y que contiene las siguientes columnas para cada una de ellas: título, año, géneros, keywords, director, 3 intérpretes principales, votos, idioma, país, media de votos, duración y presupuesto.

\section{Funcionamiento del motor}

El motor partirá de una película dada por el usuario, a partir de la cual se buscan las más similares y le son recomendadas al usuario. La elección de las películas para recomendar se realizará en dos etapas diferenciadas:
\begin{enumerate}
    \item Elegir $N$ películas en base a una medida de similaridad, de forma que tengan un contenido parecido a la entrada dada por el usuario.
    \item Seleccionar las $M, M<N$ películas más populares de entre esas $N$ películas.
\end{enumerate}

Esta estrategia podría ser discutida y probablemente existan estrategias ligeramente superiores. Sin embargo, se ha escogido esta porque conceptualmente parece lógico que aunque se realice una preselección de $N \sim 30$ películas basándose en el contenido de las mismas y la similaridad en términos de keywords, se haga un segundo filtrado. Por lo general, las películas más populares serán mejores recomendaciones para los usuarios (siempre habrá usuarios mas snobs, pero lo que le gusta a mucha gente es probable que le guste al usuario). Es por eso que de la primera preselección de películas que se hace con las keywords, posteriormente se filtran con criterios más manuales. Es aquí donde se introduce ese conocimiento del dominio que tan importante es para obtener resultados buenos.


\subsection{Similaridad}

El primer paso en la construcción del motor de recomendación consiste en definir un criterio que aporte información sobre cómo de parecidas son dos películas. Partiendo de la película seleccionada por el usuario, se toma el director, los actores principales, los géneros y las keywords de la película y se crea una matriz como la mostrada en la \autoref{tab:similarity}.

\begin{table}[H]
\centering
\begin{tabular}{|c|c|c|c|c|c|c|c|c|c|c|}
\hline
\textbf{\begin{tabular}[c]{@{}l@{}}titulo\\ película\end{tabular}} &
  \textbf{director} &
  \textbf{actor 1} &
  \textbf{a2} &
  \textbf{a3} &
  \textbf{keyword 1} &
  \textbf{k2} &
  \textbf{género1} &
  \textbf{g2} &
  \textbf{...} &
  \textbf{gk} \\ \hline
Película 1  & $a_{11}$ & $a_{12}$ &  &  & ...      &  &  &  & ... & $a_{1q}$ \\ \hline
...    &          &          &  &  & ...      &  &  &  & ... &          \\ \hline
Película i & $a_{i1}$ & $a_{i2}$ &  &  & $a_{ij}$ &  &  &  & ... & $a_{iq}$ \\ \hline
...    &          &          &  &  & ...      &  &  &  & ... &          \\ \hline
Película p & $a_{p1}$ & $a_{p2}$ &  &  & ...      &  &  &  & ... & $a_{pq}$ \\ \hline
\end{tabular}
\caption{Matriz generada para el cálculo de la similaridad entre dos películas. Se utiliza la información del director, los actores principales, las keywords que aparecen y los géneros a los que pertenece la película. .}
\label{tab:similarity}
\end{table}

En esta matriz, el elemento $ij$ toma el valor $0,1$ dependiendo de la correspondencia entre la película $i$ y el contenido de la columna $j$ (que depende de la película seleccionada). Por ejemplo, si ``keyword1'' está en la película $i$, tendremos $a_{ij} = 1$ y $0$ en otro caso. Una vez definida la matriz, se determina la distancia entre dos películas mediante

\begin{eqnarray}
d_{m, n} = \sqrt{  \sum_{i = 1}^{N} \left( a_{m,i}  - a_{n,i} \right)^2  } 
\end{eqnarray}

En este punto, únicamente tenemos que seleccionar las $N$ películas que son más cercanas a la entrada seleccionada por el usuario.

\subsection{Popularidad}

Atendiendo a la similaridad entre películas, seleccionamos una lista de $N$ películas. En este punto, seleccionaremos únicamente 5 películas. Para ello, damos una puntuación a cada entrada. Se computa la puntuación de acuerdo a estos tres criterios:
- La puntuación media.
- El siguiente ratio:
\begin{equation}
    R = \frac{\frac{votes - \bar{votes}}{\sigma_{votes}}}{\frac{gross - \bar{gross}}{\sigma_{gross}}}
\end{equation}

Que mide cuántos usuarios votaron la película respecto al presupuesto de la película, ambos datos normalizados.
- El año de lanzamiento.

Los dos primeros serán una medida directa de la popularidad de varias entradas. Para el tercer criterio, se introduce el año de lanzamiento. Se asume que las preferidas por la persona serán en la mayoría de los casos de la misma época.

A continuación, calculamos la puntuación de acuerdo a esta ecuación:

\begin{equation}
\mathrm{score} = puntuacion^2 \times \phi_{\sigma_1, c_1} \times  \phi_{\sigma_2, c_2},
\label{eq:criteria}
\end{equation}

donde $\phi$ es una función gaussiana del tipo

\begin{equation}
\phi_{\sigma, c}(x) \propto \mathrm{exp}\left(-\frac{(x-c)^2}{2 \, \sigma^2}\right).
\end{equation}

En el caso del año se centra la gaussiana en el año de lanzamiento de la película elegida por el usuario y se elige $\sigma = 10$.\\

Además, como se verá más adelante, se realizarán algunas operaciones más con el fin de mejorar los resultados obtenidos.

\section{Definición de las funciones del motor}

En primer lugar es necesario extraer de la entrada seleccionada por el usuario, para ello se implementa la siguiente función:
\begin{lstlisting}[language=Python, caption= {Variables utilizadas para la creación de la matriz de coordenadas.}, label= {lst:variablesfila}]
def variables_entrada(df, id_entrada): 
    """Calcula los valores tomados por las variables director_name, actor_[1,2,3]_name y plot_keywords para la
    película seleccionada por el usuario.
    Args:
        df (pd.DataFrame): DataFrame de películas
        id_entrada (int): Id de la entrada seleccionada
    Returns:
        list: Lista que contiene los valores extraidos para la película seleccionada
    """
    
    etiquetas_col = []    
    if pd.notnull(df['director_name'].iloc[id_entrada]):
        for s in df['director_name'].iloc[id_entrada].split('|'):
            etiquetas_col.append(s)
            
    for i in range(3):
        columna = 'actor_NUM_name'.replace('NUM', str(i+1))
        if pd.notnull(df[columna].iloc[id_entrada]):
            for s in df[columna].iloc[id_entrada].split('|'):
                etiquetas_col.append(s)
                
    if pd.notnull(df['plot_keywords'].iloc[id_entrada]):
        for s in df['plot_keywords'].iloc[id_entrada].split('|'):
            etiquetas_col.append(s)
    return etiquetas_col
\end{lstlisting}

Una vez se tienen las variables con las que se calculará la distancia entre la película seleccionada por el usuario y las demás, se genera la matriz de $0$ y $1$ con una película por fila dependiendo de si coincide o no la información de ambas películas:

\begin{lstlisting}[language=Python, caption= {Creación de la matriz de coordenadas de las películas. Para ello, se hace uso de la función definida en el \autoref{lst:variablesfila}. Además, se añaden los valores de los géneros de la película.}]
def anade_variables(df, VAR_REF):
    """Añade al dataframe de películas las columnas dadas en VAR_REF (que serán el 
    director, etc de unapelícula) y las inicializa a 0 o 1 dependiendo de si la 
    película es del mismo director, tiene a ese actor, etc
    Args:
        df (pd.DataFrame): DataFrame de películas
        VAR_REF (list): Salida de aplicar variables_entrada sobre el df y una película
    Returns:
        pd.DataFrame: DataFrame con las nuevas películas
    """
    for s in VAR_REF: 
        df[s] = pd.Series([0 for _ in range(len(df))])
    columnas = ['genres', 'actor_1_name', 'actor_2_name',
                'actor_3_name', 'director_name', 'plot_keywords']
    for categoria in columnas:
        for indice, fila in df.iterrows():
            if pd.isnull(fila[categoria]):
                continue
            for s in fila[categoria].split('|'):
                if s in VAR_REF: df.at[indice, s] = 1            
    return df
\end{lstlisting}

Con estas dos funciones ya definidas, podemos crear la función \texttt{recomienda}, que será la que realice la selección de las $N$ películas más cercanas a la dada por el usuario. La métrica utilizada para evaluar cuales son las películas más cercanas es la distancia euclídea.

\begin{lstlisting}[language=Python, caption= {Obtención de las 30 mejores películas dada una por el usuario.}]
def recomienda(df, id_entrada, N = 31):
    """Crea una lista de N películas similares a las seleccionadas por el usuario
    Args:
        df (pd.DataFrame): DataFrame de películas
        id_entrada (int): Id de la entrada seleccionada
        N (int, optional): Number of films recommended 
        (take into account that the nearest will be always itself). Por defecto 31.
    Returns:
        list: List of ids of films recommended
    """
    df_copia = df.copy(deep = True)    
    lista_generos = set()
    for s in df['genres'].str.split('|').values:
        lista_generos = lista_generos.union(set(s))    

    # Creación de variables adicionales para comprobar la similaridad
    variables = variables_entrada(df_copia, id_entrada)
    variables += list(lista_generos)
    df_nuevo = anade_variables(df_copia, variables)

    # Determinación de los vecinos más próximos: la distancia se calcula con las nuevas vairables
    X = df_nuevo[variables].values
    nbrs = NearestNeighbors(n_neighbors=N, algorithm='auto', metric='euclidean').fit(X)
    
    xtest = df_nuevo.iloc[id_entrada][variables].values
    xtest = xtest.reshape(1, -1)

    _, indices = nbrs.kneighbors(xtest)

    return indices[0][:]
\end{lstlisting}

Una vez se tienen las $N$ mejores películas, como se ha dicho anteriormente, se ordenan de forma descendente de acuerdo al criterio de la \autoref{eq:criteria} (implementado en el \autoref{lst:criteria} utilizando la función \texttt{extrae\_parametros}):

\begin{lstlisting}[language=Python, caption={Extrae algunas variables del dataframe dado como entrada y devuelve la lista de N películas. Esta lista se ordena de acuerdo al criterio de la función criterio\_seleccion, definida en el \autoref{lst:criteria}}]
def extrae_parametros(df, ultimas_peliculas, N = 31):
    """Extrae algunas variables del dataframe dado como entrada y devuelve la lista de N películas. Esta lista se ordena de acuerdo al criterio de la función criterio_seleccion.
    Args:
        df ([type]): DataFrame de películas
        ultimas_peliculas (list): Lista con las n películas recomendadas
        N (int, optional): Number of films recommended. Por defecto 31.
    Returns:
        list: Películas recomendadas
    """
    parametros_peliculas = ['_' for _ in range(N)]
    i = 0
    maximo_usuarios = -1
    for indice in ultimas_peliculas:
        parametros_peliculas[i] = list(df.iloc[indice][['movie_title', 'title_year',
                                        'vote_average', 
                                        'num_voted_users', 'gross']])
        parametros_peliculas[i].append(indice)
        maximo_usuarios = max(maximo_usuarios, parametros_peliculas[i][4] )
        i += 1
    # The first element is the selected film itself
    titulo_principal = parametros_peliculas[0][0]
    ano_ref  = parametros_peliculas[0][1]
    
    parametros_peliculas.sort(key = lambda x:criterio_seleccion(titulo_principal,
                                                            ano_ref, 
                                                            titulo = x[0], 
                                                            ano = x[1],
                                                            puntuacion = x[2],
                              votos_norm = ((x[3] - df.num_voted_users.mean())/df.num_voted_users.std()),
                              ganancias_norm = ((x[4] - df.gross.mean())/df.gross.std())),
                                                            reverse = True)
    
    return parametros_peliculas
\end{lstlisting}


\begin{lstlisting}[language=Python, caption= {Criterios de selección para la elección de las mejores películas. Se parte de la puntuación de la película y se modifica según la proximidad temporal de la película a evaluar con la película de referencia y el ratio entre votos e ingresos de la película.}, label={lst:criteria}]
def criterio_seleccion(titulo_principal, ano_ref, titulo, ano, puntuacion, votos_norm, ganancias_norm):
    """Calcula la puntuación de una película como recomendación de otra en base a la similaridad de su título, la distancia temporal entre ambos lanzamientos y el número de votos de la película evaluada y la puntuación de la película. Además, la similitud entre títulos se tiene en cuenta para evitar la recomendación de secuelas. Es decir, si dos películas tienen un nombre muy similar, se desechara como recomendación.
    Args:
        titulo_principal (str): Título de la película dada por el usuario
        ano_ref (int): Año de lanzamiento de la película dada por el usuario
        titulo (str): Título de la película a evaluar
        ano (int): Año de lanzamiento de la película a evaluar
        puntuacion (float): Votación media de la película a evaluar
        ratio (float): Ratio entre votos y presupuesto de la película
    Returns:
        float: Nota de la película dada
    """
    ratio = votos_norm/ganancias_norm
    if pd.notnull(ano_ref):
        factor_1 = filtro_gaussiano(ano_ref, ano, 10)
    else:
        factor_1 = 1        
    if pd.notnull(ratio):
        factor_2 = filtro_gaussiano(ratio, 0, 1)
    else:
        factor_2 = 0
    if secuela(titulo_principal, titulo):
        mark = 0
    else:
        mark = puntuacion * factor_1 * factor_2
    return mark
\end{lstlisting}

Llegados a este punto, supóngase que el usuario ha elegido la película Harry Potter y la Piedra Filosofal. Sin ejecutar el recomendador y sin sonar demasiado raro, parece que las películas que recomendará el sistema, si no se modifica lo dicho hasta ahora, serán las demás películas de Harry Potter, ya que tendrán las mismas keywords o muy parecidas, compartirán actores y serán de la misma época. Sin embargo, estas películas sean probablemente conocidas por el usuario. Por tanto, se implementa la función \texttt{sequela}, que trata de ver, mediante lógica difusa y medidas como la distancia Levenshtein \cite{levenshtein}, que mide la distancia entre dos palabras en términos de los cambios necesarios para llegar de una a otra. La función \texttt{secuela}, definida en el \autoref{lst:sequel} servirá para filtrar los resultados dados al usuario.

\begin{lstlisting}[language=Python, label={lst:sequel},caption={Determinación de si dos películas son o no secuelas y eliminarlas. Se determina que dos películas son secuelas si tienen el mismo título o es muy similar.}]
def secuela(titulo_1, titulo_2):   
    """Compara los títulos de dos películas y devuelve si son similares o no
    Args:
        titulo_1 (str): Primer título
        titulo_2 (str): Segundo título
    Returns:
        bool: True cuando las películas son secuelas. False en otro caso
    """
    if fuzz.ratio(titulo_1.lower(), titulo_2.lower()) > 50 or fuzz.token_set_ratio(titulo_1.lower(), titulo_2.lower()) > 60:
        return True
    else:
        return False
def quita_secuelas(seleccion_peliculas):
    """Removes sequels from the list of films given
    Args:
        seleccion_peliculas (list): Lista de películas de la que quitar las secuelas
    Returns:
        list: Lista sin secuelas
    """ 
    quitadas_de_seleccion = []
    for i, pelicula_1 in enumerate(seleccion_peliculas):
        for j, pelicula_2 in enumerate(seleccion_peliculas):
            if j <= i: continue 
            if secuela(pelicula_1[0], pelicula_2[0]): 
                last_film = pelicula_2[0] if pelicula_1[1] < pelicula_2[1] else pelicula_1[0]
                quitadas_de_seleccion.append(last_film)
    lista_peliculas = [film for film in seleccion_peliculas if film[0] not in quitadas_de_seleccion]
    return lista_peliculas
\end{lstlisting}

Finalmente, se implementa la función que se encarga de obtener estas 5 mejores películas:

\begin{lstlisting}[language=Python, caption={Función que se encarga de completar la selección de las 5 películas que se recomendarán.}]
def add_a_seleccion(seleccion_peliculas, parametros_peliculas, N = 31):
    """Completa la lista seleccion_peliculas que contiene 5 películas que se recomendarán al usuario. Las películas son seleccionadas de parametros_películas y sólo se tienen en cuenta si el título es suficientemente distinto del de otras películas.
    Args:
        seleccion_peliculas (list): Lista de películas
        parametros_peliculas (list): Lista de parámetros
        N (int, optional): Películas a puntuar. Por defecto 31.
        M (int, optional): Películas a recomendar. Por defecto 5.
    Returns:
        list: peliculas seleccionadas
    """
    lista_peliculas = seleccion_peliculas[:]
    icuenta = len(lista_peliculas)    
    for i in range(N):
        ya_en_lista = False
        for s in seleccion_peliculas:
            if s[0] == parametros_peliculas[i][0]: 
                ya_en_lista = True
            if secuela(parametros_peliculas[i][0], s[0]): 
                ya_en_lista = True            
        if ya_en_lista: continue
            
        icuenta += 1
        if icuenta <= 5:
            lista_peliculas.append(parametros_peliculas[i])
    return lista_peliculas
\end{lstlisting}

A modo de \textit{wrapper} de este flujo, la función \textit{find\_similarities} ejecuta todo el \textit{pipeline} de búsqueda de las películas recomendadas. Por ejemplo, para la película de Piratas del Caribe, el sistema realiza las siguientes recomendaciones:

\begin{enumerate}
    \item La comunidad del anillo
    \item Harry Potter y la cámara secreta
    \item X-Men: Días del futuro pasado
    \item El Hobbit, un viaje inesperado
    \item King Kong
\end{enumerate}

\begin{lstlisting}[language=Python, caption={Wrapper del pipeline de búsqueda de recomendaciones. Se trata de la función final, es decir, la que llama a todas las demás para encontrar las películas recomendadas. Parte de un ID de película y busca las similares en términos de keywords, géneros, etc., las ordena por la heurística definida y devuelve las seleccionadas}]
def encuentra_similaridades(df, id_entrada, borra_secuelas = True, verbose = False, N = 31):
    """Dado el id de una película busca las 5 mejores recomendaciones.
    Args:
        df (pd.DataFrame): [description]
        id_entrada (int): [description]
        borra_secuelas (bool, optional): Borrar secuelas de las recomendaciones. Por defecto True.
        N (int, optional): Películas a evaluar. Por defecto 31.
        M (int, optional): Películas a recomendar. Por defecto 31.
    Returns:
        list: Selección de películas recomendadas
    """
    if verbose: 
        print(90*'_' + '\n' + "QUERY: films similar to id={} -> '{}'".format(id_entrada,
                                df.iloc[id_entrada]['movie_title']))
    ultimas_peliculas = recomienda(df, id_entrada, N)
    # Crear lista de N películas
    parametros_peliculas = extrae_parametros(df, ultimas_peliculas, N)
    # Seleccionar 5 películas de la  listaSelect 5 films from this list
    seleccion_peliculas = []
    seleccion_peliculas = add_a_seleccion(seleccion_peliculas, parametros_peliculas, N)
    # Borrado de las secuelas
    if borra_secuelas: 
        seleccion_peliculas = quita_secuelas(seleccion_peliculas)
    # Añadir nuevas películas a la lista
    seleccion_peliculas = add_a_seleccion(seleccion_peliculas, parametros_peliculas, N)
    señeccion_titulos = []
    for i,s in enumerate(seleccion_peliculas):
        señeccion_titulos.append([s[0].replace(u'\xa0', u''), s[4]])
        if verbose: print("nº{:<2}     -> {:<30}".format(i+1, s[0]))
    return señeccion_titulos
\end{lstlisting}

\section{Resumen}

En este capítulo, uno de los más importantes del proyecto, se ha creado el sistema de recomendación en sí, creando la medida de similaridad entre películas y realizando la selección de las mejores y se ha explicado el motivo de la inclusión de cada criterio de selección. Además, se ha mostrado un resultado del sistema a modo de ejemplo. La parte de explotación del resultado será tratada en el siguiente capítulo, en el que se introducirá el sistema en una API, abriendo la posibilidad de desplegar la solución en un entorno productivo utilizable por usuarios reales.